% !TEX program = lualatex
% !TEX spellcheck = it_IT
% !TEX root = TM.tex

\documentclass[ structure = article
              , maketitlestyle = standard
              , liststyle = aligned
              %, headerstyle = center
              , secfont = roman
              , secstyle = center
              ]{suftesi}

\usepackage[no-math]{fontspec}
\usepackage[rm,mono=false]{libertine}
\usepackage{polyglossia}
\setmainlanguage{italian}
\usepackage{microtype}
\usepackage[hidelinks,breaklinks]{hyperref}
\usepackage[italian=quotes]{csquotes}
\newcommand\q\enquote

\usepackage{imakeidx}
\indexsetup{level=\section*}

\usepackage{libertinust1math}
\usepackage{MnSymbol,mathtools}
\usepackage[bb=ams]{mathalfa}
%\usepackage[bb=ams,cal=rsfs]{mathalfa}

\usepackage{amsthm}
\newcounter{coun}
\counterwithin{coun}{section}
\theoremstyle{definition}
\newtheorem{definizione}[coun]{Definizione}
\newtheorem{lemma}[coun]{Lemma}
\newtheorem{proposizione}[coun]{Proposizione}
\newtheorem{corollario}[coun]{Corollario}
\newtheorem{osservazione}[coun]{Osservazione}
\newtheorem{esempio}[coun]{Esempio}
\newtheorem{esercizio}[coun]{Esercizio}

%\usepackage{calc}

\renewcommand\theequation{\thesection.\arabic{equation}}

\newcommand\nil\varnothing
\renewcommand\bar\overline
\newcommand\set[1]{\left\{#1\right\}}
\newcommand\abs[1]{\left\lvert#1\right\rvert}
\newcommand\enne{\mathbb N}
\newcommand\erre{\mathbb R}

\newcommand\nota[1]{{\color{red}[#1]}}
