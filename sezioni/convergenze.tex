% !TEX program = lualatex
% !TEX spellcheck = it_IT
% !TEX root = ../TM.tex

\section{Convergenze}

\begin{definizione}[Convergenza quasi uniforme]
Sia \((\Omega, \mathcal A, \mu)\) uno spazio di misura. Diciamo che  una successione di funzioni misurabili \(\set{f_n : \Omega \to \erre \mid n \in \enne}\) {\em converge quasi uniformemente} a \(f : \Omega \to \erre\) qualora per ogni \(\epsilon > 0\) esiste un \(E \in \mathcal A\) per cui \(\mu(\Omega \setminus E) < \epsilon\) e \(f_n \to f\) uniformemente su \(E\) per \(n \to +\infty\).
\end{definizione}

La convergenza quasi uniforme implica quella quasi ovunque.

\begin{proposizione}
Sia \((\Omega, \mathcal A, \mu)\) uno spazio di misura e \(\set{f_n : \Omega \to \erre \mid n \in \enne}\) una successione di funzioni misurabili. Se converge quasi uniformemente a \(f : \Omega \to \erre\), allora converge quasi ovunque a \(f\), cioè esiste un \(N \in \mathcal A\) di misura nulla per cui
\[\lim_{n \to +\infty} f_n(x) = f(x) \quad \text{per ogni } x \in \Omega \setminus N .\] 
\end{proposizione}

\begin{proof}
Sia \(\varepsilon > 0\) qualsiasi. Contestualmente abbiamo un \(E_\varepsilon \in \mathcal A\) per cui \(\mu(\Omega \setminus E_\varepsilon) < \varepsilon\) e su sul quale si ha che \(f_n \to f\) uniformemente. Se indichiamo con \(C \subseteq \Omega\) l'insieme dei punti in cui la successione converge puntualmente, abbiamo che \(E_\varepsilon \subseteq C\), e quindi \(\mu(\Omega \setminus C) \le \mu(\Omega \setminus E_\varepsilon) < \varepsilon\). Per l'arbitrarietà di \(\varepsilon\), possiamo concludere che \(\mu(\Omega \setminus C) = 0\) e possiamo scegliere \(N = \Omega \setminus C\).  
\end{proof}

\begin{esempio}
Consideriamo \(f_n := \chi_{[n, n+1]}\). Converge puntualmente a \(0\) in ogni punto di \(\erre\), per ogni insieme \(E \subseteq \erre\) di misura finita, si ha
\[\sup_{x \in \Omega \setminus E} \abs{f_n(x)} = 1\]
da un certo \(n\) in poi.
\end{esempio} 

Come abbiamo appena visto, l'implicazione contraria non è sempre vera. Tuttavia sotto un'ipotesi aggiuntiva la cosa funziona.

\begin{proposizione}[Teorema di Severini-Egorov]\label{proposizione:SeveriniEgorov}\index{teorema!di Severini-Egorov}
Sia \((\Omega, \mathcal A, \mu)\) uno spazio di misura con \(\mu(\Omega) < +\infty\) e \(\set{f_n : \Omega \to \erre \mid n \in \enne}\) una successione di funzioni misurabili. Se converge quasi ovunque a \(f : \Omega \to \erre\), allora converge quasi uniformemente a \(f\). 
\end{proposizione}

Prima di partire con la dimostrazione, è essenziale per la dimostrazione che segue osservare che sono equivalenti questi due fatti:
\begin{enumerate}
\item \(f : \Omega \to \erre\) converge puntualmente verso \(f : \Omega \to \erre\)
\item per ogni \(\delta > 0\) si ha
\[\Omega = \bigcup_{m \in \enne} \bigcap_{n \in \enne, n \ge m} \set{x \in \Omega \mid \abs{f_n(x) - f(x)} < \delta} .\]
\end{enumerate}
La Teoria della Misura subentra nel momento in cui ci si rende conto che che al variare di \(m \in \enne\) al successione degli insiemi misurabili
\[\bigcap_{n \in \enne, n \ge m} \set{x \in \Omega \mid \abs{f_n(x) - f(x)} < \delta}\]
è crescente. e che quindi si può usare il Lemma~\ref{lemma:LimitOfMeasures}.

\begin{proof}[Dimostrazione della Proposizione~\ref{proposizione:SeveriniEgorov}]
Possiamo supporre che \(f_n\) converga puntualmente per ogni \(x \in \Omega\), questo senza ledere la validità della dimostrazione: infatti se \(f_n\) non converge puntualmente su tutto \(\Omega\), allora basta rimuovere il sottoinsieme di \(\Omega\) per cui non lo fa, essendo questo di misura nulla.\newline
Introduciamo dei nomi: per \(k, n \in \enne\)
\[E_{n, k} := \set{x \in \Omega \left\mid \abs{f_n(x) - f(x)} < \frac1k \right.}\]
Come abbiamo visto prima di iniziare la dimostrazione, abbiamo che per ogni \(k \in \enne\), gli \(E_{n, k}\) al variare di \(n \in \enne\) ricoprono \(\Omega\).
%\[\Omega = \bigcup_{m \in \enne} \bigcap_{n \ge m} E_{n, k} .\]
Inoltre \(\set{E_{n,k} \mid n \in \enne} \in \mathcal A\) è una successione crescente e quindi per il Lemma~\ref{lemma:LimitOfMeasures} abbiamo che
\[\mu(\Omega) = \lim_{m \to +\infty} \mu \left(\bigcap_{n \ge m} E_{n, k}\right) .\]
Di conseguenza, poiché \(\mu(\Omega) < +\infty\), per ogni \(k \in \enne\) possiamo scegliere \(m_k \in \enne\) in modo che
\[\mu \left( \Omega \setminus \bigcap_{n \ge m_k} E_{n, k} \right) < \frac{\epsilon}{2^k} .\]
Abbiamo quindi
\[\mu \left( \Omega \setminus \bigcap_{k \in \enne} \bigcap_{n \ge m_k} E_{n, k} \right) \le \sum_{k \in \enne} \mu \left(\Omega \setminus \bigcap_{n \ge m_k} E_{n, k} \right) = \epsilon\]
e per costruzione \(f_n\) converge uniformemente verso \(f\) su \(\displaystyle \bigcap_{k \in \enne} \bigcap_{n \ge m_k} E_{n, k}\).
\end{proof}

\begin{definizione}[Convergenza in misura]
Sia \((\Omega, \mathcal A, \mu)\) uno spazio di misura. Diciamo che una successione di funzioni misurabili \(\set{f_n : \Omega \to \erre \mid n \in \enne}\) {\em converge in misura} a \(f : \Omega \to \erre\) qualora per ogni \(\epsilon > 0\) si ha
\[\lim_{n \to +\infty} \mu \left( \set{x \in \Omega \mid \abs{f_n(x) - f(x)} \ge \varepsilon} \right) = 0 .\]
\end{definizione}

\begin{proposizione}
Sia \((\Omega, \mathcal A, \mu)\) uno spazio di misura e \(\set{f_n : \Omega \to \erre \mid n \in \enne}\) una successione di funzioni misurabili. Se \(f_n\) converge a \(f : \Omega \to \erre\) quasi uniformemente, allora \(f_n\) converge a \(f\) in misura.
\end{proposizione}

\begin{proof}
Sia \(\varepsilon >0\) qualsiasi. Allora abbiamo \(A_\varepsilon \in \mathcal A\) tale che \(\mu(\Omega \setminus A_\varepsilon) < \varepsilon\) e \(f_n \to f\) uniformemente. Quindi esiste un certo \(n_\varepsilon \in \enne\) tale che
\[A_\varepsilon \subseteq \set{x \in \Omega \mid \abs{f_n(x) - f(x)} < \varepsilon} \text{ per ogni } n \in \enne, n \ge n_\varepsilon .\]
Quindi
\[\mu\underbrace{\left(\Omega \setminus \set{x \in \Omega \mid \abs{f_n(x) - f(x)} < \varepsilon} \right)}_{\set{x \in \Omega \mid \abs{f_n(x) - f(x)} \ge \varepsilon}} \le \mu\left(\Omega \setminus A_\varepsilon\right) < \varepsilon. \qedhere\]
\end{proof}

In generale, invece, non accade il contrario.

\begin{esempio}
Nella famiglia degli intervalli del tipo
\[\left[\frac{i}{j}, \frac{i+1}{j}\right] \text{ per } j \in \enne_{\ge 1}, i \in \enne, i \le j\]
definiamo una successione di insiemi in questo modo: \(A_0 := [0, 1]\) e poi, avendo \(A_n := \left[\frac{i}{j}, \frac{i+1}{j}\right]\) per certi \(i, j\),
\[A_{n+1} := \begin{cases} \left[0, \frac{1}{j+1}\right] & \text{se } i+1=j \\ \left[\frac{i+1}{j}, \frac{i+2}{j}\right] & \text{altrimenti} \end{cases}\]
Ora esaminiamo \(f_n := \chi_{A_n}\). Osserviamo che
\[\mu \left( \set{x \in \Omega \mid \abs{f_n(x) - f(x)} \ge \varepsilon}\right) = \nil\]
se \(\varepsilon > 1\), quindi limitiamoci soltanto a \(0 < \varepsilon \le 1\). In questo caso poi
\[\mu \left( \set{x \in \Omega \mid \abs{f_n(x) - f(x)} \ge \varepsilon}\right) = A_n .\]
La lunghezza degli intervalli \(A_n\) è decrescente e per ogni \(\delta > 0\) si può sempre trovare \(n \in \enne\) tale che \(\mu(A_n) \le \delta\). Questo ci permette di concludere che \(f_n \to 0\) in misura. Tuttavia non converge quai ovunque a \(0\), a dire il vero il limite puntuale non è definito in alcun punto. Di conseguenza non può convergere neanche quasi uniformemente.
\end{esempio}

\begin{lemma}[di Borel-Cantelli]\label{lemma:BorelCantelli}\index{lemma!di Borel-Cantelli}
Sia \((\Omega, \mathcal A, \mu)\) uno spazio di misura e una qualsiasi successione \(\set{A_n \mid n \in \enne} \subseteq \mathcal A\). Se
\[\sum_{n \in \enne} \mu\left(A_n\right) < +\infty\]
allora
\[\mu\left( \bigcap_{m \in \enne} \bigcup_{n \ge m} A_n \right) = 0 .\]
\end{lemma}

\begin{proof}
Poiché \(\displaystyle\mu\left( \bigcup_{n \in \enne} A_n \right) < +\infty\), possiamo applicare il Lemma~\ref{lemma:LimitOfMeasures} alla successione decrescente \(\displaystyle \set{\left.\bigcup_{n \ge m} A_n \right\mid m \in \enne}\) .
\end{proof}

\begin{proposizione}
Sia \((\Omega, \mathcal A, \mu)\) uno spazio di misura e \(\set{f_n : \Omega \to \erre \mid n \in \enne}\) una successione di funzioni misurabili convergenti in misura verso \(f : \Omega \to \erre\). Allora \(\set{f_n \mid n \in \enne}\) ha una sottosuccessione convergente verso \(f\) quasi ovunque.
\end{proposizione}

\begin{proof}
Per ogni \(k \in \enne\) possiamo scegliere un \(n_k \in \enne\) per cui
\[\mu \left(\set{x \in \Omega \left\mid \abs{f_{n_k}(x) - f(x)} \ge \frac1k \right. } \right) < \frac{1}{k^2} .\]
Possiamo pertanto concludere applicando il Lemma~\ref{lemma:BorelCantelli}: infatti l'insieme
\[\bigcap_{k \in \enne} \bigcup_{n \ge k} \set{x \in \Omega \left\mid \abs{f_{n_k}(x) - f(x)} \ge \frac1k \right. }\]
ha misura nulla e sul complementare \(f_{n_k}\) converge puntualmente contro \(f\).
\end{proof}
